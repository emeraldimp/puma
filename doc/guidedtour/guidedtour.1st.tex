\documentclass[12pt]{article}
\usepackage{fullpage, graphicx, url}
\setlength{\parskip}{1ex}
\setlength{\parindent}{0ex}
\title{PUMA 3.1 Guided Tour}
\begin{document}
\section*{Welcome to PUMA 3.1!}


PUMA 3.1 is a combination wiki and content management system. By now, you've installed the database and your site/ directory, and are ready to begin creating content. This document will take you on a guided tour through PUMA, including page creation, editing, and uploading files.


 To begin, load up the main index page.
 
 \includegraphics[scale=0.75]{images/tour1.png} 


 As you can see, the base site/ I'm using is for Creole West Productions, who have generously sponsored development of PUMA. Your home page will look different, of course.
\subsection*{Logging In}


 Right now, we can't do anything, not even edit the main page! This is because we haven't logged in yet. Let's do that now.


 Change index.php to user.php in the browser's address bar, which will bring up the login page. PUMA, if you used the default SQL schema, comes with a default user, admin. The password is `password'. So, let's log in. 
 
 \includegraphics[scale=0.75]{images/tour2.png} 


 Once you have logged in, you'll be taken to your user preferences page. Go ahead and return to the HomePage.


 You'll notice that we now have some new options on the right side of the page.
 
 \includegraphics[scale=0.75]{images/tour3.png} 


 First, we're told when the page was last edited (1). Below that, we have two access settings. PUMA has four different access levels: Anybody, Users, Editors and Admin. The first dropbox (2), to the right of the spectacles, determines who can view the page, while the second detemines who can edit the page (3). Note that, although you can set the page to be editable by Anyone, only users who are logged in may actually edit it. To the right of that is a wand (4), which allows us to set the permissions.


 NOTE: You cannot set permissions the first time you edit a page!
\subsection*{Editing a Page}


 Let's create a HomePage. The final button is the `Edit Page' button. Click it now.


 This will bring up the PUMA Editor, which acts just like a regular text editor with some PUMA-specific additions. For now, type in some sample text, and click `Preview Changes', which will show how the text will look on the page. Since this is just an example, go ahead and click `Save Changes'.

\includegraphics[scale=0.75]{images/tour4.png} 

 If you had made some changes, but decided that you didn't like them after all, you could click `Cancel Changes' to leave the editor and return to the page you were editing.


 We now have some more options on the right side of the page. We can see when the page was last edited, by whom, and which revision this is. We can also view the history of the page, by clicking on the book (1), or which pages link to the current page (2). We'll come back to these in a minute.

\includegraphics[scale=0.75]{images/tour5.png} 

\subsection*{The PagePicker, PUMA Plugins, and creating a linkbar}


 Let's first put some links in the linkbar, to make navigation easier. Replace `index.php' (and everything that follows it) with `edit.php?page=HeaderLink'. This will bring up the edit page for the link bar. Now we can use some of the extras in the editor.


 Click on the `PagePicker' link. \includegraphics[scale=0.75]{images/tour6.png}  This will bring up PUMA's PagePicker, which can locate created pages and automatically insert the correct link for you. Since we only have the HomePage, we'll go ahead and insert a link there. Type `home' in the search window, which will bring up suggested pages.

\includegraphics[scale=0.75]{images/tour7.png} 

 Click on `HomePage', which will automatically fill in the rest of the options for you. We can change what the link text will look like in the next box. Let's change it to just `Home'. You'll see the preview automatically update below. We can now click on `OK' to insert the link, or cancel to exit out. Click OK.


 Now, press space, and then click on the `PumaPlugin' link. \includegraphics[scale=0.75]{images/tour8.png}  This brings up the PumaPlugin dialog, with a list of all the available Puma Plugins. We want to add a link that will allow users to login and out, so select `User Login Link' and click OK.

\includegraphics[scale=0.75]{images/tour9.png} 

 The editor will show a graphic where the link will go, but a quick `Preview Changes' will show how it's replaced with, in our case, `admin'. This will change depending on the user, or, if no user is logged in, it will say, `Login'.


 Let's see how to create a new page. This will be the ever-important `Contact Information' page. Click on the PagePicker button again, and this time type `Contact' in the first box. You can see that the PagePicker can't find any relevant pages, so it offers to create a link to a new page. Go ahead and click on `Contact'. Everything else looks OK, so click `OK' to insert the link.


 Now, save the HeaderLink page. The Linkbar has been updated! 
 
 \includegraphics[scale=0.75]{images/tour10.png} 
\subsection*{Creating a New Page}


 Creating a new page is as easy as editing it, literally. Click the `Contact' link and then edit the page. Put in some contact information.


 The editor also has a built-in character map and international keyboard. \includegraphics[scale=0.75]{images/tour11.png}  They function fairly standardly, and will not be discussed further here.

\includegraphics[scale=0.75]{images/tour11a.png} 

 As a warning, some functions cannot be performed on the first edit of a page, such as the ResourcePicker and ResourceUploader, demonstrated below.


 Save the contact page, and then return to the HomePage. Edit it.
\subsection*{Resources}


 Suppose that we wanted to add a picture to the home page (or any other page, for that matter). How could we do that? We use the ResourceUploader and the ResourcePicker. \includegraphics[scale=0.75]{images/tour12.png}  Click on the ResourceUploader first.


 The nickname is a short name to describe the resource—which can be nearly any type of file, not just an image—that you are uploading. Multiple files can have the same nickname; it is solely for your use. The description is for a better description. You can click browse to pick out your file on the disk. When you're ready, click `Upload' to upload the file. If the upload is successful, you will get a message saying so.

\includegraphics[scale=0.75]{images/tour13.png} 

 You can upload more files or click `Cancel' to return to the editor.


 Let's insert the image we just uploaded. Click on the ResourcePicker. The Resource Picker will find all resources associated with a particular page so you can attach them to the page. Images are displayed in the page itself; others are linked.


 You can select the resource you want from the dropdown.

\includegraphics[scale=0.75]{images/tour14.png} 

 The description will automatically change based on which resource you have chosen. Select the desired resource, and click OK. The ResourcePicker inserts the appropriate code into the editor.


 Go ahead and save the changes to see the result. As you can see, the image has been inserted.

\includegraphics[scale=0.75]{images/tour15.png} \subsection*{Examining Revisions}

 Now, we can see the difference between two versions of a page by clicking on the `Compare to last' button on the right (1). \includegraphics[scale=0.75]{images/tour16.png}  This will give a detailed report of the most recent changes. (see below) We can also click on the `History' button (2) to see a list of all revisions, and go back to any previous version.

\includegraphics[scale=0.75]{images/tour17.png} 

\includegraphics[scale=0.75]{images/tour18.png} 

 Finally, we should log out of PUMA for security. Click on the `admin' link in the linkbar to go to the user page. From there, we can click log out.

\end{document}
